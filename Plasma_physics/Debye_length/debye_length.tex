%This is my super simple Real Analysis Homework template

\documentclass{article}
\usepackage[margin=1in]{geometry}
\usepackage[utf8]{inputenc}
\usepackage[english]{babel}
\usepackage[]{amsthm} %lets us use \begin{proof}
\usepackage[]{amssymb} %gives us the character \varnothing
\usepackage{amsmath}

\title{Debye length}
\author{John ``Jack'' Brooks}
\date\today
%This information doesn't actually show up on your document unless you use the maketitle command below

\begin{document}
	
	
\maketitle %This command prints the title based on information entered above
	


\section{Debye length derivation}
The Debye length is the scale length associated with plasma screening out external electric fields, $\mathbf{E}$.  To show this, we start with Gauss's law and convert $\mathbf{E}$ to potential, $\phi$ with $\mathbf{E} = - \nabla \phi$.  

\begin{equation}
\begin{split}
	%\label{eq:asdf}
	\nabla^2 \phi &= \frac{1}{\epsilon_0}\left( \rho_{plasma} \right) \\
	&= \frac{1}{\epsilon_0}\left[ q n_0 \, \textrm{exp} \left( \frac{-q \phi (r)}{k_b T} \right)\right] 
\end{split}
\end{equation}
where and the plasma charge is assuming a Boltzmann distribution.  

If we assume that $k_bT \gg q\phi$, then we can Taylor expand the exponential term to get
\begin{equation}
\begin{split}
%\label{eq:asdf}
\nabla^2 \phi = \frac{1}{\epsilon_0}\left[ q n_0 \,  \left(1- \frac{q \phi (r)}{k_b T} \right)\right]
\end{split}
\end{equation}
Assuming that we have equal ion and electron charge density, $q_e n_{oe}$ cancels with $q_i n_{oi}$ leaving
\begin{equation}
\begin{split}
%\label{eq:asdf}
\nabla^2 \phi = \frac{1}{\epsilon_0}\left[ -   \frac{q^2 n_0 \phi (r)}{k_b T}  \right].
\end{split}
\end{equation}
The solution to this is 
\begin{equation}
\begin{split}
\phi &= \phi_0 \, \textrm{exp} \left( -r / \left( \frac{q^2n_0}{\epsilon_0 k_b T} \right)^{1/2} \right) \\
 &= \phi_0 \, \textrm{exp} \left( -r /\lambda_D^{1/2} \right)
\end{split}
\end{equation}
where the scale length, 
\begin{equation}
\begin{split}
 \lambda_D =  \left( \frac{q^2n_0}{\epsilon_0 k_b T} \right)^{1/2},
\end{split}
\end{equation}
is called the Debye length.

\section{Debye length screening}

If we now include a test charge, $Q_{tc}$, to our Gauss's law equation,
\begin{equation}
\begin{split}
%\label{eq:asdf}
\nabla^2 \phi &= \frac{1}{\epsilon_0}\left( \rho_{plasma} - \rho_{tc}\right) \\
&= \frac{1}{\lambda_D^2}\phi - \frac{Q_{tc}}{\epsilon_0}\delta (r-r_0)
\end{split}
\end{equation}
we can now investigate how the plasma screens out this charge.

TODO

\end{document}


